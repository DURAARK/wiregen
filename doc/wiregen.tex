\documentclass[12pt]{scrartcl}
 
\usepackage{ucs}
\usepackage[utf8x]{inputenc}
\usepackage[T1]{fontenc}
\usepackage[ngerman]{babel}


\title{WireGen}
\author{Ulrich Krispel}
\date{\today{}, Graz}

\begin{document}
 
\maketitle

\section{Introduction}

Creating a power line hypothesis: why and how

\section{Electrical Appliances Installation}

Norms do specify "preferred zones" for installation of power lines. These are dependant on the position of openings in the wall (doors, windows), as well as the type of building (office, residential).

\section{Formulating a Installation Zone Grammar}

- attributed grammar
- TODO: non context-free rules for identifying 2-3-4 sets of socket/switch into one element
- installation zone grammar evaluation results in graph
- marked "endpoints" for 

\section{Cable Structure Hypothesis}

\begin{itemize}
\item given the installation zone graph, hypothesize a connection between endpoints
\item assumption minimizing cost = minimizing used material for power lines
\item hypothesis: try to connect endpoints to a given "root", preferrably using installation zones
\item results in the minimum steiner tree problem, steiner tree in graphs: Given an edge weighted graph $G = (V, E, c)$ and a subset $P \subseteq V$. The  Steiner tree problem is finding the tree in $G$ that spans all vertices of $S$ with a minimum edge cost. 
\item This problem is known to be NP-complete \cite{Hwang1992}.
\item Approximation algorithm, e.g. \cite{Kou1981}, or an own approximation: 
\end{itemize}

\bibliographystyle{alpha}
\bibliography{bib}
 
\end{document}